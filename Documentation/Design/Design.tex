\documentclass[12pt]{article}
\usepackage[margin=1in]{geometry}
\usepackage{setspace}
\onehalfspacing

\usepackage{floatrow}

\usepackage{multirow}
\usepackage{algorithm}
\usepackage{algorithmic}
\usepackage{array}
\usepackage{graphicx}

\begin{document}

\title{n8 Game Framework Design Document}
\author{Nate Ebel}

\maketitle

\tableofcontents
\pagebreak

\section{Design and Use Patterns}
\subsection{Command}
Object oriented replacement for callbacks that encapsulates a request within an object.  Benefits and uses include:
\begin{itemize}
	\item Allows easy redo/undo functionality if do and undo methods are implemented and some type of history is kept
	\item Can be made general enough to operate on different entities such as a ``jump'' command that works for both user controlled and computer controlled entities
	\item Can become a sandbox if many common operations are included with the base class
	\item If command is stateless, a single instance of the command may be used for the entire program
	\item Can be easily mapped to new inputs
\end{itemize}

\subsection{Observer}
Allows passing of information between loosely coupled components through the use of notify, and onNotify methods.  Objects can be either subjects that notify others when certain events happen, or they can be observers and respond to received notifications from observed objects.  Observers register an interest in an object so it may receive notifications from that object.  This pattern allows dissimilar objects to know about each other.  For example, a GUI object may want to update a life meter when a user entity is hit by an enemy.  The user class doesn't need to know about the GUI object, it just needs to broadcast that its life total has changed.

\subsection{Subclass Sandbox}
\subsection{Singleton}
\subsection{Service Locator}
\subsection{Components}

\section{	Client Use/Interaction}
Client should instantiate a single instance of a Game object.  

\section{ Resource Management}

\section{ Game Loop}
Takes place within Game and uses the Game object's GameTimer instance to control the loop.  Within the loop the current GameState is processed at each iteration.
		
\section{Window Management}
Should be a window object.  This object should be stored by the Game class.  Should support renaming and resizing.  
	
\section{	Game States}
A GameState is whatever is happening in the forefront of the game.  GameStates will be maintained on a stack within a GameStateManger.  GameStateManager acts as a bridge between the game loop within Game and the currently active GameState.  Should support passing of entities between states, and each GameState will maintain a list of its entities.  GameStates will include 3 key methods of responding to user input, updating, and rendering.  The Update method will receive the current time as an argument, and Render will receive the current window canvas as an argument.  These 3 methods will be called in the game loop within the Game object.  Pause and resume methods can be overridden to handle changes to and from the GameState.

\section{	Entities}

\section{	Components}

\section{	Entity States}

\section{	Input}

\section{	Events}

\section{	Rendering}
\subsection{Rendering operations}
\begin{itemize}
	\item Render single entity
	\item Render single entity with a camera object (Camera: see Section \ref{sec:Camera})
\end{itemize}
\subsection{Static sprite}
\subsection{Animated sprite}


\section{ Camera} \label{sec:Camera}

\section{	Operations on or using entities}
		AI
		physics

\section{	Game levels}

\section{ Globally Available Services}
Use globally available service locator.  The service locator should support:
\begin{itemize}
	\item registering and unregistering of services
	\item should be initialized at start.  This is when service objects will be provided to the locator
\end{itemize}

\subsection{ Logging }
Logging should be available to any aspect of the system. 
\subsection{ Audio Player} 

\section{	System wide values and enums}

\section{	Configuration data/files}

\section{Use Cases}
\subsection{Input changes game state}
\subsection{Input changes player state}
walking $\rightarrow$ flying
\subsection{Input moves player}
\subsection{Entity destroyed and animation is played}
\subsection{Timed event activated}
\subsection{Timed event destroyed}
\subsection{Entity collision}
\subsection{AI player must make a decision}
\subsection{Notify an achievement system when 5 enemies are destroyed}
\subsection{Remap an action's key}




\end{document}
